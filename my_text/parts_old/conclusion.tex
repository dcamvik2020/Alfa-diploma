
\section{Заключение}
В рамках проведённой работы были уточнены детали устройства метода рендеринга с использованием иерархических атласов, предложен ряд усовершенствование в алгоритме отображения, а также разработано комплексное программное решение имплементирующее метод. Разработанное решение позволяет добиться значительного прироста в качестве картинки по сравнению с наиболее распространённым методом, уровнями детализации. Производительность решения удовлетворительна на этапе предобработки и сравнима с наивным подходом на этапе рендеринга.

В силу большой модульности и сложности метода предобработки, при дальнейших исследованиях планируется повышать производительность и качество практически всех его частей. Планируется реализовать некоторое количество альтернативных подходов и сравнить получаемое качество, а также время предобработки. Среди прочего планируются исследования в направлении поддержки неоднородных вершинных атрибутов посредством проведения границ кластеров по границам разрывов, ускорение этапа репараметризации используя иные алгоритмы оптимизации, использование иных методов кластеризации, а также внедрение модификаций предлагаемых в \cite{feng2010feature}. Планируется пересмотр этапа распрямления границ. В результате этой операции эвристика планарности кластеров сильно ухудшается, что в некоторых ситуациях приводит к недостаточной частоте семплирования выступов моделей. Более тонкий учёт геометрии поверхности на этапе сглаживания может помочь избавится от этой проблемы. Отметим, что аналогичный эффект проявляется и на этапе квадрангуляции, но в меньшей степени.

Ещё одним крупным направлением для дальнейших исследований являются алгоритмы адаптации при рендеринге иерархического атласа. Также планируется внедрение параллелизма в текущую имплементацию рендеринга (вынесение обновления виртуального кэша в отдельный поток), и наконец рассмотрение возможности перенесения вычислений нагружающих центральный процессор в текущей имплементации (адаптация, вычисление мип уровней для границ) на графический процессор.

По мнению автора подход виртуализации весьма перспективен в применении к современным приложениям реального времени, так как способен удовлетворить всё растущие требования к визуальному качеству, а также позволяет более эффективно использовать ресурсы GPU, а поэтому заслуживает дальнейшего изучения и усовершенствования.
