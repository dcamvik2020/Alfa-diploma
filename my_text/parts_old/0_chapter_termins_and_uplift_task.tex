\section{Введение}
\label{sec:intro}


\iffalse

\subsection{Ключевые термины и обозначения}
\subsection{Бизнес-проблема}
\subsection{Область исследований}

\fi


При проведении разного рода маркетинговых кампаний бизнесу важно понять, на какую группу клиентов она возымеет наиболее сильный эффект. Например, в телекоме или банковском секторе это может напрямую влиять на затраченную сумму денег на проведение соответствующей кампании, а в результате и на финальную прибыль. Для этого важно уметь достаточно точно определять группу клиентов, которые, например, не купят продукт без скидки, но при этом со скидкой они его купят. Таких клиентов называют чувствительными, и нахождение таких клиентов позволит не отправлять им лишнюю таргетированную рекламу, а также не предоставлять скидку тем клиентам, которым она недостаточно интересна. Аналогичный вопрос касательно воздействий на объекты наблюдается и в медицине: при создании нового лекарства важно, а иногда это может быть и просто необходимо, выделить группу пациентов, для которых лекарство вызывает различные побочные эффекты (насколько выше шансы, что человек приобретёт побочное заболевание или побочный эффект, приняв наше лекарство). Также может быть полезно понять, для каких пациентов это лекарство действует наиболее эффективно (насколько выше шансы, что человек выздоровеет, приняв наше лекарство). Таким образом, может получиться разработать раличные лекарства под различные группы пациентов, при этом имеющие максимальный эффект для этих групп.

\par
Такого рода задачи требуют отдельного подхода к моделированию, который называется uplift-моделированием. Само понятие uplift связано с ростом какой-то характеристики: например, с увеличением шансов (вероятности, склонности) совершить некоторое целевое действие. Целевым действием может быть покупка некоторого продукта, взятие или выдача кредита наличными в банке, избавление от заболевания. При этом достичь роста этой вероятности можно с помощью какого-то одного определённого воздействия или целой группы воздействий на объект, под которым можно понимать клиента компании. Например, в маркетинге воздействием может быть показ рекламы или предоставление скидки на продаваемый товар в явном (например, клиент видит надпись о скидке на купоне в магазине или на странице с товаром в интернете) или неявном (клиент сразу видит изменённую с учётом скидки цену, без указания о её наличии) виде. И здесь возникает фундаментальная проблема: невозможно одновременно повоздействовать и не повоздействовать на один и тот же объект. Например, невозможно одновременно и дать, и не дать клиенту скидку на некоторый продукт -- это взаимоисключающие события. Поэтому напрямую моделировать эффект от воздействия невозможно, так как для этого необходимо знать исходы в обоих случаях. Это значит, что невозможно напрямую определить, является ли клиент чувствительным к воздействию или нет. Но можно попробовать оценить средний эффект для некоторой группы похожих в некотором смысле объектов. При этом нужно разделть эту подгруппу объектов на две группы, в одной из которых было применено воздействие, а в другой это воздействие применено не было. Только тогда получится оценить некоторый средний эффект от воздействия, под которым тогда будет пониматься величина, на которую возросла средняя конверсия в тестовой группе относительно контрольной группы. Под средней конверсией здесь понимается доля объектов, для которых целевое действие совершилось (произошло целевое событие, клиент купил продукт). При этом тестовой группой называется группа с воздействием, а контрольной группой называется группа без воздействия. Таким образом, если знать средние склонности к целевому действию в этих двух группах, то получится оценить разницу между ними, то есть получится оценить эффект от воздействия.



\iffalse

проблема
возможно решение
основные понятия

\fi



\par
В машинном обучении известны методы, работающие с группами объектов и умеющие выделять подгруппы схожих объектов. Это так называемые ''деревянные'' модели: решающее дерево (decision tree), случайный лес (random forest) и градиентный бустинг (gradient boosting). На данный момент в том или ином виде адаптировать под решение задачи uplift-моделирования получилось их все, о чём будет рассказано более подробно далее в данной работе. Но несмотря на заметное количество разработанных методов uplift-моделирования, отбора признаков и метрик для оценки качества моделей, некоторые вопросы ещё слабо освещены. Например, на практике часто случается так, что uplift-модели, несмотря на то, что по построению они должны находить чувствительных клиентов, находят склонных клиентов, что выражается в том, что для более склонных клиентов рост конверсии при наличии воздействия больше, чем у тех клиентов, кто изначально склонен к целевому действию меньше. Причин у этой проблемы может быть множество: не достаточно аккуратно собранные данные, не совсем подходящий метод отбора признаков для моделирования, не подходящий конкретно для этой задачи метод моделирования, неудачно выбранная целевая метрика для оптимизации.

\par
Исходя из вышесказанного, данная работа ставит своей целью анализ того, насколько возможно уйти от склонностоной зависимости эффекта от воздействия, применяя различные методы отбора признаков для банковских данных большой размерности (порядка 2000 признаков).



\iffalse


% ----------------------- ТЕМА РАБОТЫ ------------------------
		анализ возможности ухода от склонностной зависимости предсказаний
								в аплифт моделях
		текущая тема норм
			отбор признаков = предмет исследования
			нужно исследовать и рост качества в зависимости от метода
			это и есть исследование -- вот это дало такое качество
% ------------------------------------------------------------


данные -- идея отсекать моделью склонности слишком склонных и несклонных (из относительного аплифта -- как дальнейшие шаги и в экспериментах тоже)


их задачей по определению является, влияние отборов специализированные отборы признаков для uplift-моделирования и их влияние на качество финальных моделей. Также мало внимания 


\fi







\iffalse
 
 
 Например, для мета-алгоритма S-learner, о котором подробнее пойдет речь далее в работе, для интерпретации важностей использованных признаков можно использовать встроенные в модель методы вычисления feature\_importances\_. Это позволит получить некоторое представление о том, какие характеристики имеют наибольшее значение при прогнозе эффекта от воздействия. Например, это может дать понимание о процессе принятии решения о предоставлении клиенту скидки, то есть описать некоторый \textbf{\underline{портрет}} чувствительного к скидке клиента, который может перестать пользоваться продуктом или просто его не купить без предоставления ему такой скидки. Такое понимание важно для бизнеса, поскольку поможет понимать, почему было принято то или иное решение, а не просто полагаться на число из ''черного ящика''.

\par
Особое место в процессе решения задачи uplift-моделирования, как и при решении любой другой задачи машинного обучения, занимает отбор признаков. Эта стадия позволит убрать из рассмотрения лишние характеристики объектов, которые затрудняют понимание процесса или просто не помогают принять правильное решение. Также это выгодно с точки зрения вычислительной мощности: при использовании 1000 признаков тратится куда больше ресурсов, чем при использовании 20 признаков. Также это может влиять на структуру модели, что может, в свою очередь, повлечь увеличение времени обучения модели, работы финального алгоритма, обработки данных (препроцессинг, преобразование, ...)




\par
Uplift-моделирование ставит своей целью измерить эффект от некоторого воздействия на некоторый объект. Этот эффект проявляется в изменении значения целевой переменной при неизменном признаковом описании объекта. Например, целевой переменной может быть факт выздоровления человека, тогда воздействием может быть приём пациентом лекарства, а отсутствие воздействия в таком случае можно представить как приём пациентом ''таблетки-пустышки'', и тогда эффект лекарства будем измеряться относительно ситуации, когда имеет место лишь эффект плацебо. Но воздействия бывают и других типов, задачи uplift-моделирования также востребованы и в бизнесе: например, измерение чувствительности клиента телекоммуникационной компании к повышению тарифа на некоторый процент или чувствительность клиента банка к повышению процентной ставки. Обычно рассматривают при этом всё же не повышение, а понижение цены товара, поскольку тогда чувствительными клиентами окажутся те, кто купит продукт лишь со скидкой, то есть такая постановка задачи позволит удовлетворить потребности большего числа клиентов или даже расширить клиентскую базу.

\par
Слово \textbf{\underline{чувствительность}} здесь использовано не просто так: эта характеристика позволяет понять, как сильно изменится \textbf{\underline{склонность}} клиента к покупке продукта. Склонность к покупе можно воспринимать и моделировать как вероятность покупки. Таким образом, чувствительность может показать, насколько эта вероятность изменится при наличии воздействия (скидки) относительно ситуации без воздействия (продажа по обычной цене). При этом важно понимать, что сильно чувствительным клиентом является клиент с низкой склонностью без воздействия и с высокой склонностью с воздействием. Если клиент изначально достаточно сильно хочет купить некоторый продукт, то с точки зрения бизнеса нет оснований предлагать скидку такому клиенту. То есть цель uplift-моделирования найти чувствительных, но не склонных клиентов.

\par
Сама постановка задачи отличается от классической постановки задачи бинарной классификации (например, определения склонности клиента к чему-то). Поэтому использовать модели-классификаторы в uplift-моделировании не стоит -- есть специальные классы моделей для решения подобных задач, о которых и будет идти речь. Но при этом до сих пор не так много внимания уделялось анализу признаков, которые характеризуют чувствительных клиентов. Например, как описать клиента банка, чувствительного к понижению процентной ставки при рефинансировании? Какой клиент согласится на базовое предложение со ставкой $X\%$, а какой согласится лиьш на ставку $(X-\delta)\%$? На подобные вопросы касательно \textbf{\underline{портрета}} чувствительного клиента и хочется ответить в ходе данной работы.

\par 
Целью данной работы является исследование различных способов отбора признаков для задачи upllift-моделирования в случае большой размерности данных (около 2000 признаков) и сравнение uplift-моделей, обученных на сжатых признаковых описаниях, с базовым решением -- моделью склонности, обученной на признаковом описании, полученном одним из классических методов отбора признаков (например, по значению feature importance, которое считают большинство моделей на основе решающих деревьев).















\fi


\subsection{Обзор литературы}
\label{sec:related}
Рассмотрим несколько перспективных подходов к адаптивному рендерингу и сравним их с выбранным в этой работе.


\subsubsection{Иерархический атлас}
Сравнивая метод представленный в \cite{niski2007multi} 