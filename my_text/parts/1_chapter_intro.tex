\section{Введение}













\subsection{Актуальность исследования}

При проведении разного рода маркетинговых кампаний бизнесу крайне важно сократить расходы на них, при этом оставить эффективность коммуникации с клиентами на максимально возможно уровне для сохранения их лояльности. Например, в телекоме или банковском секторе это может напрямую влиять на затраченную сумму денег на проведение соответствующей кампании, а в результате и на финальную прибыль. Для этого важно уметь достаточно точно определять группу клиентов, которые, например, не купят продукт без скидки, но при этом со скидкой купят. Нахождение таких клиентов позволит не отправлять таргетированную рекламу в телекоме и не предоставлять в ритейле скидку тем клиентам, которым она не так интересна. ''Похожая'' ситуация наблюдается и в медицине: при создании нового лекарства может быть важно выделить группу пациентов, на которых это лекарство действует наиболее эффективно. Таким образом, получится разработать разные лекарства, имеющие максимальный эффект для различных групп пациентов, на основе имеющихся данных. Одним из многообещающих подходов для решения подобных задач является uplift-моделирование, которое позволяет оценить эффект от воздействия на клиента и тем самым помогает принимать наиболее эффективные для бизнеса решения.

Из-за большой практической важности решения данного типа задач, интерес к этой области сохраняется на протяжении более 30 лет. Было разработано множество методов моделирования эффекта от воздействий, оценки качества построенных моделей. Также были предложены некоторые методы отбора признаков, используемых для построения моделей. Существует большое количество исследований на тему отбора признаков для моделей машинного обучения, решающих классические задачи машинного обучения, такие как классификация и регрессия. Однако, несмотря на это методы отбора признаков для такой специфической задачи, как задача uplift-моделирования, изучены слабо. Но данные методы особенно важны в случае больших данных, поскольку число и качество используемых в финальной модели переменных напрямую влияет на её качество, стабильность качества во времени и способность подстроиться под изменения распределений основных используемых признаков.

Таким образом, данное исследование имеет достаточно большую степень актуальности. Оно может принести пользу при принятии эффективных решений на основе имеющихся данных в крупном бизнесе при проведении различных маркетинговых кампаний или оценке эффекта от воздействий других типов.
















\subsection{Цели и задачи работы}

Данная работа ставит своей целью исследовать влияние методов отбора признаков на качество uplift-моделей. При этом будет рассматриваться банковская задача определения чувствительности клиентов к изменению процентной ставки при рефинансировании кредитов наличными. Данная задача позволяет провести полный анализ качества полученных моделей и выделенных признаков с точки зрения их интерпретируемости и информативности. Для достижения поставленной цели необходимо проанализировать не только качество полученных моделей с точки зрения некоторых метрик для фиксированного значений порога и кривых теоретического характера, показывающих степень ранжирующей способности моделей. Необходимо также оценить визуально, насколько хорошо модель выделяет ту небольшую часть чувствительных клиентов, которую мы ищем. Для этого используются диаграммы, строящиеся по разбиению группы клиентов на равные сегменты на основе их средней чувствительности к воздействию. Также для понимания эффекта от отбора признаков необходимо сравнить полученные модели с бейзлайном (так называют базовую модель, с которой сравнивают все остальные) в виде модели склонности (так часто называют бинарный классификатор). Для этого нужно сравнить то, как они сортируют клиентов по чувствительности и насколько их предсказания коррелируют в целом.















\subsection{Методология исследования}

В данном исследовании для достижения поставленной цели будет использоваться методология, состоящая из нескольких отдельных стадий.

Сперва нужно определить и формально поставить решаемую задачи с учётом банковской специфики. Это означает, что нужно учесть необходимость интерпретации полученных результатов и используемых методов отбора признаков. Нужно чётко описать требования к решению поставленной задачи.

Далее нужно понять, как будут сравниваться построенные модели и как будет оцениваться эффект от использования различных методов отбора признаков. Для этого необходимо определить критерии или метрики качества моделей, причём как числовые (они помогут быстро получить ответ на вопрос, получилось ли улучшить бейзлайн или нет), так и графические (помогут визуально собрать значительно больше информации о качестве конкретной модели). Однако, из-за большого числа анализируемых комбинаций моделей и методов отбора признаков, визуальный анализ стоит применять лишь для сравнения некоторых моделей: например, для сравнения лучшей модели с худшей и базовой моделью склонности.

Затем идёт стадия сбора данных для данной задачи и их предварительные обработка и анализ: необходимо собрать выборку клиентов за определённый период и все имеющиеся для них признаки, затем очистить данные от дублей, выбросов и определить, как обработать пропуски и категориальные переменные.

Когда все описанные выше стадии пройдены, можно перейти к построению базовой uplift-модели на всех имеющихся после удаления неинформативных признаков данных и базовой модели склонности на отобранных для неё одним из методов отбора признаков для классической задачи бинарной классификации. Это поможет оценить, какое качество можно получить в принципе, без использования методик для снижения размерности данных и отбора информативных признаковых описаний. Стоит учитывать при этом, что изначально в данных могут быть неинформативные и даже коррелирующие признаки, которые могут снизить качество модели. Это означает, что также одной из целей отбора признаков является понять, насколько можно улучшить качество базовой uplift-модели.

После построения базовых моделей нужно проанализировать существующие методы моделирования и отбора признаков и проверсти эксперименты с их использованием.

В конце нужно провести подробный анализ полученных результатов и сформулировать выводы. Это позволит понять, насколько использованные методы отбора признаков применимы в решении банковских задач и какой эффект они имеют на модель относительно бейзлайна.

Данный план поможет четко отследить эффект от применияемых методов отбора признаков с разных сторон как с помощью числовых метрик, так и с помощью визуализаций.





















\subsection{Структура работы}

Данная работа состоит из семи глав.

Первая глава содержит введение, в котором описаны актуальность исследования, цель работы, а также методология исследования и структура работы.

Во вторая глава описаны теоретические основам uplift-моделирования, основные концепции, лежащие в основе существующих методов решения данного типа задач, а также сами методы моделирования и оценки качества построенных моделей. Также указаны и области применения uplift-моделирования.

В третье главе содержится обзор существующих методов отбора признаков для классических задач машинного обучения, таких как классификация и регрессия. В ней рассматриваются различные подходы к отбору признаков, включая методы фильтрации, встраивания и обёртки.

Четвертая глава описывает методы отбора признаков, специфичные для uplift-моделирования, а также их отличия от классических методов отбора признаков, применимых в классических задачах машинного обучения.

В пятой главе описана экспериментальная часть работы: описаны используемые данные, их предварительная обработка, схема построения базовых моделей склонности и uplift-модели на всех имеющихся после обработки данных признаках, а также применение методов отбора признаков, использование предзказаний моделей для ранжирования клиентов и оценка качества полученных предсказаний.

Шестая глава содежрит достаточно полный анализ влияния отбора признаков на качество моделей. В частности, описаны зависимости предсказаний модели склонности и лучшей uplift-модели.

Седьмая глава содержит заключение. В нём описываются основные выводы и результаты исследования, обсуждаются перспективы дальнейшей работы.




